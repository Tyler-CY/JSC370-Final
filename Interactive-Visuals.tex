% Options for packages loaded elsewhere
\PassOptionsToPackage{unicode}{hyperref}
\PassOptionsToPackage{hyphens}{url}
%
\documentclass[
]{article}
\usepackage{amsmath,amssymb}
\usepackage{lmodern}
\usepackage{iftex}
\ifPDFTeX
  \usepackage[T1]{fontenc}
  \usepackage[utf8]{inputenc}
  \usepackage{textcomp} % provide euro and other symbols
\else % if luatex or xetex
  \usepackage{unicode-math}
  \defaultfontfeatures{Scale=MatchLowercase}
  \defaultfontfeatures[\rmfamily]{Ligatures=TeX,Scale=1}
\fi
% Use upquote if available, for straight quotes in verbatim environments
\IfFileExists{upquote.sty}{\usepackage{upquote}}{}
\IfFileExists{microtype.sty}{% use microtype if available
  \usepackage[]{microtype}
  \UseMicrotypeSet[protrusion]{basicmath} % disable protrusion for tt fonts
}{}
\makeatletter
\@ifundefined{KOMAClassName}{% if non-KOMA class
  \IfFileExists{parskip.sty}{%
    \usepackage{parskip}
  }{% else
    \setlength{\parindent}{0pt}
    \setlength{\parskip}{6pt plus 2pt minus 1pt}}
}{% if KOMA class
  \KOMAoptions{parskip=half}}
\makeatother
\usepackage{xcolor}
\usepackage[margin=1in]{geometry}
\usepackage{graphicx}
\makeatletter
\def\maxwidth{\ifdim\Gin@nat@width>\linewidth\linewidth\else\Gin@nat@width\fi}
\def\maxheight{\ifdim\Gin@nat@height>\textheight\textheight\else\Gin@nat@height\fi}
\makeatother
% Scale images if necessary, so that they will not overflow the page
% margins by default, and it is still possible to overwrite the defaults
% using explicit options in \includegraphics[width, height, ...]{}
\setkeys{Gin}{width=\maxwidth,height=\maxheight,keepaspectratio}
% Set default figure placement to htbp
\makeatletter
\def\fps@figure{htbp}
\makeatother
\setlength{\emergencystretch}{3em} % prevent overfull lines
\providecommand{\tightlist}{%
  \setlength{\itemsep}{0pt}\setlength{\parskip}{0pt}}
\setcounter{secnumdepth}{-\maxdimen} % remove section numbering
\usepackage{booktabs}
\usepackage{caption}
\usepackage{longtable}
\ifLuaTeX
  \usepackage{selnolig}  % disable illegal ligatures
\fi
\IfFileExists{bookmark.sty}{\usepackage{bookmark}}{\usepackage{hyperref}}
\IfFileExists{xurl.sty}{\usepackage{xurl}}{} % add URL line breaks if available
\urlstyle{same} % disable monospaced font for URLs
\hypersetup{
  pdftitle={Interactive Visuals},
  hidelinks,
  pdfcreator={LaTeX via pandoc}}

\title{Interactive Visuals}
\author{}
\date{\vspace{-2.5em}}

\begin{document}
\maketitle

\hypertarget{major-crime-indicator-offences-by-toronto-neighbourhoods}{%
\section{Major Crime Indicator Offences by Toronto
Neighbourhoods}\label{major-crime-indicator-offences-by-toronto-neighbourhoods}}

Below is an interactive heat map of number of MCI offences by
neighbourhood.

The crime map shows the distribution of MCI offences in the 140
offences, filtered by offence type and total count. From the heatmap, we
could see that areas which were further away from city centers and urban
areas had lower number of reported offences.

\hypertarget{major-crime-indicator-offences-per-day-by-type}{%
\section{Major Crime Indicator Offences per Day by
Type}\label{major-crime-indicator-offences-per-day-by-type}}

We also inspected the trend of offence occurrences with respect to time
in 2016 with the histogram below. The trend shown in the plot suggests
that there were no significant relationship between the day of year and
the number of major crime occurred, as the plot appears to be close to a
uniform distribution with only minor fluctuations periodically. The
spike at Day 0 is due to the fact the some data with missing date were
marked with 0, but these were kept in our cleaned data since we cared
more about the number of occurrences instead of crimes at a specific
date.

\end{document}
